\begin{Large}
	\centertext{Abstract}
\end{Large}

\vskip 0.1in
\noindent The is SPIRIT a narrow band ultra-low power RF transceiver, intended for RF wireless applications in the sub-1 GHz band. It is designed to operate in both the license-free ISM and SRD frequency bands at 433,868 and 920 MHz, but can also be programmed to operate at other additional frequencies in the 430-470 MHz, 860-940 MHz bands. It comes under the category of linux-wpan.\\
 Linux-wpan is mnaged by the Alexander Aring, Pengutronix. This technology is fit for exchanging the data over a short distance communications and also have many benefits over the existing system such as Bluetooth and linux-wan. \\ 
\noindent Linux, a Unix-like and mostly POSIX –compliant, open source operating system kernel is also the most used kernel in operating systems varying from PC’s, mobiles, SOC’s, etc.\\
\noindent This project report documents in detail the integration of SPIRIT, A ultra-low power transceiver by STMicroelectronics with the Linux Kernel. A detailed account is presented of the study of SPIRIT protocol Stack, Linux device model, and all other things that contributed to the final product i.e; a Linux device driver for SPIRIT.
